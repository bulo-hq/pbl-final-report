\chapter{課題解決のプロセスの詳細}

\section{各人の課題の概要とプロジェクト内における位置づけ}
%各人の担当課題の概要と、プロジェクト内における役割・位置づけを記述する。

未来花子の担当課題は以下のとおりである。 
\begin{description}
 \item[4月] Webからのレシピ収集・データベース化 。
 \item[5月] レシピの内容のグループ分け。
 \item[6月] 特産品**を含むレシピ検索。
 \item[7--9月]特産品**を含むレシピ考案。
\end{description}

北海花子の担当課題は以下のとおりである。 
\begin{description}
 \item[4月] 草むしり。
 \item[5月] 畑仕事。
 \item[6月] 庭弄り。
\end{description}

\bunseki{未来}

\section{担当課題解決過程の詳細}
%各人の担当課題の解決過程を詳細に記述する新規習得技術を必ず含むこと。

\subsection{未来太郎}
\begin{description}
 \item[4月] Webからのレシピ収集・データベース化 
Webの検索機能を用いて、レシピを検索した。 
材料と手順について、データベースを作成した。 
データベース化には、講義「**」で習得した**の手法を用いた。 
その際、**の技術を習得した。
 \item[5月]まずは草むしりからはじめた。
\end{description}

\bunseki{未来太郎}


\subsection{北海花子}

省略。
\bunseki{北海花子}


\section{担当課題と他の課題の連携内容}
%各人の担当課題とプロジェクト内の他の課題との連携について記述する。

\subsection{未来花子}
自分はWebからのレシピ収集を行ったが、そのデータベースは他の分野から収集した https://ja.overleaf.com/project/5f0efd508f2e5f00015c9b4f
レシピをも入力するものである。データベース構築の際、**の部分を担当した。
以下略。
\bunseki{未来花子}

\subsection{北海花子}
省略。
\bunseki{北海花子}
