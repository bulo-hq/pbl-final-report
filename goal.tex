\chapter{到達目標}

\begin{figure}[htbp]
 \centering
 \includegraphics[width=3cm]{images/apple.eps}
 \caption{リンゴ}
 \label{fig:apple}
\end{figure}

\section{本プロジェクトにおける目的}\label{sec:mokuteki}
%1.3節で述べた課題をより具体的に記述する。成果に対して必ず満たすべき条件を含む

地域の特色を生かしたおいしいカレーの作り方が課題。
最終的には、100人中75人以上がおいしいというカレーの詳細なレシピを作ること。
またそのカレーは函館の海産物を用いたものであること。
レシピの手順は、なぜその方法がいいのかも含めて記述されていること。
\bunseki{未来}

\subsection{通常の授業ではなく、プロジェクト学習で行う利点}

本課題では材料に多種多様なものが考えられるが、複数の人数で試作することにより、 
さまざまなバリエーションが試せる。また、味の好みの偏りが少なくなる。 
通常の授業では基本的に個人の知識・技術について講義・演習形式で行われるため、 
共同作業で行うべき作業時間の多いテーマに関しては向かない。

\bunseki{函館}

\subsection{地域との関連性(必要ならば)}

海産物を特徴にしたカレーができると地域の名物料理として売り出せるかも。 
また函館の特産品の売り上げが伸びるかも。

\bunseki{北海}


\section{具体的な手順・課題設定}\label{sec:tejun}
%2.1節で述べた課題を解決するための小課題を手順に分け、その詳細を記述する。 各人への割り当て可能なレベルまで具体化する。なお、以下を必ず含むこと。 
%・このような課題設定に至るプロセス
%・各小課題の解決過程に関連する講義 
%・各小課題の解決過程で用いる既存技術、また習得技術

数多いレシピをもとに効率的に函館特産物を用いたカレーを製作し、コスト面から、 
試作の数を20種類以内に収める目的で、情報収集に力をいれ、以下のように 
手順を設定した。

\begin{enumerate}
\item 従来のカレーレシピ収集(料理本・テレビ・Web)
\par 課題:レシピを共通する部分と異なる部分にわけ、グループ化する。
異なる部分については、それぞれのメリットデメリットを挙げる。

\item 函館特産食品の種類の調査
\par 課題:生産高が多く、一般に特産品として知名度の高いものを調査する。
季節・標準的な値段・一般的な調理法とその調理法により引き出せる味の調査をする。

\item 函館特産食品とカレーとの組み合わせを調べる(過去のレシピの検索)
\par 課題:省略

\item 各材料の下ごしらえのレシピ化。
\par 課題:省略

\item 試作レシピパターンの決定
\par 課題:省略

\item 試作
\par 課題:省略

\item アンケート実施及び解析、改善点の発見
\par 課題:省略

\item 好評な試作パターンについてのバリエーションを設定。
\par 課題:省略

\item アンケート実施及び解析、改善点の発見
(75パーセント以上の好評価を得るまで8-9の繰り返し)
\par 課題:省略
\end{enumerate}

\bunseki{未来}

\section{課題の割り当て}
%2.2節で具体化した各小課題を誰にどのように分担したか、またその理由も含めて記述する。

各人の得意分野及び関連性、時間軸のスケジュールを基準に
以下のように割り当てた。
\bunseki{函館}
