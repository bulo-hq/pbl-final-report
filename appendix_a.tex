\chapter{使用した技術・ツール・知識}

\section{フレームワーク}
\subsection{Flutter}
モバイルアプリのフレームワークとしてFlutter\footnote{https://flutter.dev/}を用いた.Android,iOSの両者にサービスを展開する上でKotlin,Swiftそれぞれでネイティブアプリを作成する事も検討したが,グループの規模とメンバーの技術力を考慮して,クロスプラットフォーム開発が可能であるFlutterを用いることに決定した.

\section{ツール}
\subsection{Discord}
本グループの主な連絡手段としてDiscord\footnote{https://discord.com/}を使用した.プロジェクトの連絡はもちろん,日常生活で起こったことなどをDiscordで共有することによりチームワークの向上を実現することができた.

\subsection{Notion}
本グループでは情報を管理するツールとしてNotion\footnote{https://www.notion.so/}を使用した.議事録,日報,週報,プロダクトバックログをそれぞれページにまとめることにより,情報を見やすくすることができた.今後も情報の管理ツールとして使用をしていく.

\subsection{GitHub}
GitHub\footnote{https://github.com/}とはソースコードのバージョン管理システムである.このツールを用いることでチームでのアプリ開発を効率的に行うことができる.本グループではフロントエンド,バックエンドでそれぞれリポジトリを作成し,アプリの開発を進めている.今後はより一層アプリの開発を進めることとなるため,GitHubを利用していく.
\pagebreak
\subsection{Figma}
Figma\footnote{https://www.figma.com/}とは,ホームページやアプリケーションなどのワイヤーフレームを作成できるツールである.中間発表スライドの作成,プロトタイプの作成,函館バス訪問の際に使用したスライドの作成に使用した.コンポーネントやレスポンシブを意識し,ワイヤーフレームを作成することで,今後の開発がスムーズになるようにした.

\subsection{Slack}
本プロジェクトの主な連絡手段としてSlack\footnote{https://slack.com/}を使用した.本グループでは主な連絡手段としてDiscordを利用していたが,プロジェクトではSlackを利用するとなったため使用した.細かくチャンネルを分けることで情報がどこにあるかをわかりやすくすることができた.また,メンションという機能を利用することにより教員,プロジェクトメンバー,TAとそれぞれに通知が行くようにすることで効率の良い意思疎通を行うことができた.

\subsection{Miro}
Miro\footnote{https://miro.com/}はオンラインでホワイトボードを利用できるツールである.ブレインストーミングなど,メンバーで一斉に意見を書くときに使用した.Miroを使用することにより,メンバーそれぞれの意見,考えを効率よく共有することができた.現在は同じような機能を持ったFigJamを利用している.

\subsection{FigJam}
FigJam\footnote{https://www.figma.com/}はオンラインでホワイトボードを利用できるツールである.Miroと同じようなツールだが本グループではFigmaを利用しており同じサービス元であるためFigJamへと移行した.

\subsection{Google Drive}
Google Drive\footnote{https://drive.google.com}とはGoogleが提供するオンラインストレージサービスである.これを用いることにより手軽にファイルの共有を行うことができる.また昨年度のファイル履歴を見返すこともでき,資料作り,プロジェクトの進め方の参考とした.
\pagebreak
\section{マネジメント}
\subsection{リスクマネジメント}
プロジェクトを行ううえでリスクは必ず存在する.リスクマネジメント\cite{risk}はそのリスクに対してどのように対策するのかを考える.まず初めにプロジェクトメンバーを3グループに分け,それぞれのチームでメンバーが持ち寄った起こりそうなリスクをまとめた.その後それぞれのリスクに対して,そのリスクが被る被害,発生確率,影響度,脅威,対策を考えた.脅威については発生確率,影響度を掛け合わせたものであり,脅威の値が大きいほど対策するべき優先度が高いものとなる.

\chapter{活用した講義}
\section{アジャイルワークショップ}
アジャイルコーチの永瀬氏に,本プロジェクトで用いるチーム開発の手法であるアジャイル開発について講義をしていただいた.まず,本ワークショップの前に事前知識を得るため,アジャイル開発についてのビデオを視聴した.その知識をもとに永瀬氏からアジャイル開発とはなにか,アジャイル開発で用いられるスクラムとはなにかを学んだ.また,最後に講義を受けているメンバーを5人ずつのチームにし,スクラムについてのクイズ大会を行った.これにより,楽しみながらスクラムについて学ぶことができ,スクラムについての共通認識を持つことができた.

\section{フィールドワーク講座}
本プロジェクトの担当教員でもある,南部美砂子准教授,元木環准教授に講義をしていただき,フィールドワークに関する基本的な考え方,ありがちな間違いをご教授いただいた.収束的なフィールドワークでは,思い込みや認識の狭さなどを超えられないため,フィールドワークには問題を探しにいくのではないこと,また,人々の行為や相互行為には必ず理由や動機が存在し,それらを理解することをフィールドワークの目的とすることを学んだ.

\bunseki{稲田敬介}
