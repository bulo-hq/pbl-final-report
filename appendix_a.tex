\chapter{使用した技術・ツール・知識}

\section{Adobe Illustrator}
Adobe Illustrator\footnote{https://www.adobe.com/jp/products/illustrator.html}は,ベクターベースのグラフィックを作成するソフトウェアである.本グループではアプリケーションのアイコンやロゴの作成に使用した.

\section{Discord}
Discord\footnote{https://discord.com/}は,コミュニケーションツールである.本グループの主な連絡手段として使用した.プロジェクトに関わる連絡はもちろんのこと,日常生活で起こったことなどをDiscordで共有することによりチームワークの向上を実現することができた.

\section{Docker}
Docker\footnote{https://www.docker.com/}は,開発環境をコンテナとして扱うことができるツールである.本グループではサーバサイドの開発にDockerを用いた.Dockerを用いることにより,デプロイを容易に行うことができた.

\section{FigJam}
FigJam\footnote{https://www.figma.com/}はオンラインでホワイトボードを利用できるツールである.Miro\footnote{https://miro.com/}で同じような機能を使うことができるが,Figma\footnote{https://www.figma.com/}の利用を開始したため,FigJamへと移行した.

\section{Figma}
Figmaは,ホームページやアプリケーションなどのワイヤーフレームを作成できるツールである.中間発表スライドの作成,プロトタイプの作成,函館バス訪問の際に使用したスライドの作成,成果発表スライドの作成に使用した.コンポーネントやレスポンシブを意識し,ワイヤーフレームを作成することで,開発がスムーズになるようにした.

\section{Flutter}
Flutter\footnote{https://flutter.dev/}は,単一のコードベースから,複数のプラットフォームに向けたアプリケーションを構築できるフレームワークである.
UI (ユーザインタフェース) を複数のプラットフォームで統一できることなどのメリットがある.

\section{GitHub}
GitHub\footnote{https://github.com/}は,ソースコードのバージョン管理システムである.このツールを用いることでチームでのアプリ開発を効率的に行うことができる.本グループでは開発サービスでそれぞれリポジトリを作成し,アプリの開発を進めた.

\section{GitHub Projects}
GitHub Projectsは,GitHubが提供するタスク管理ツールである.GitHub ProjectsにGitHubのIssueを紐付けることにより,タスク管理を行うことができる.本グループではタスクの管理にGitHub Projectsを用いることにより,タスクの進捗状況を把握することができた.

\section{Google Cloud Platform}
Google Cloud Platform\footnote{https://cloud.google.com/}は,Googleが運営するPaaS (Platform as a Service) である.

\section{Google Drive}
Google Drive\footnote{https://drive.google.com/}は,Googleが提供するオンラインストレージサービスである.Google Driveを用いることにより手軽にファイルの共有を行うことができる.また昨年度のファイル履歴を見返すこともでき,資料作り,プロジェクトの進め方の参考とした.

\section{GraphQL}
GraphQL\footnote{https://graphql.org/}は,API用のクエリ言語であり,そのクエリを実行するためのランタイムである.
メリットは,クライアントがAPIに対して必要なリソースだけを要求することができること,可読性の高いスキーマにより開発におけるコミュニケーションコストを下げることができることである.
デメリットは,REST APIやgRPCと比較して,実行速度が遅いことである.

\section{gRPC}
gRPC\footnote{https://grpc.io/}は,ProtocolBuffers\footnote{https://protobuf.dev/}を用いて通信する,RPC (Remote Procedure Call) フレームワークである.

\section{Miro}
Miroは,オンラインでホワイトボードを利用できるツールである.ブレインストーミングなど,メンバーで一斉に意見を書くときに使用した.メンバーそれぞれの意見,考えを効率よく共有することができた.

\section{Notion}
本グループでは情報を管理するツールとしてNotion\footnote{https://www.notion.so/}を使用した.議事録,日報,週報,プロダクトバックログをそれぞれページにまとめることにより,情報を見やすくすることができた.

\section{PostgreSQL}
PostgreSQL\footnote{https://www.postgresql.org/}は,開発において使用したDBMS (Database Management System) である.
SQLを用いてデータの操作を行う,RDB (Relational Database) である.
メリットは,拡張性,信頼性が高いことである.
デメリットは,MySQLと比較して,実行速度が遅いことである.

\section{Slack}
Slack\footnote{https://slack.com/}は,コミュニケーションツールである.本グループでは主な連絡手段としてDiscordを利用していたが,本プロジェクトの主な連絡手段として使用した.細かくチャンネルを分けることで情報がどこにあるかをわかりやすくすることができた.また,メンションという機能を利用することにより教員,プロジェクトメンバー,TAとそれぞれに通知が届くようにすることで効率的な意思疎通を図ることができた.

\section{Visual Studio Code}
Visual Studio Code\footnote{https://code.visualstudio.com/}は,マイクロソフトが提供するソースコードエディタである.

\section{Xcode}
Xcode\footnote{https://developer.apple.com/jp/xcode/}は,Appleが提供する統合開発環境である.本グループではコードエディタとしてVisual Studio Codeを使用したが,iOS端末での動作確認を行う際には,エミュレータを使用するために使用した.


\section{リスクマネジメント}
プロジェクトを行ううえでリスクは必ず存在する.リスクマネジメント\cite{risk}はそのリスクに対してどのように対策するのかを考える.まず初めにプロジェクトメンバーを3グループに分け,それぞれのチームでメンバーが持ち寄った起こりそうなリスクをまとめた.その後それぞれのリスクに対して,そのリスクが被る被害,発生確率,影響度,脅威,対策を考えた.脅威については発生確率,影響度を掛け合わせたものであり,脅威の値が大きいほど対策するべき優先度が高いものとなる.

\bunseki{大津武琉}
