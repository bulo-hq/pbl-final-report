\chapter{使用した技術・ツール・知識}

\section{Adobe Illustrator}
Adobe Illustrator\footnote{https://www.adobe.com/jp/products/illustrator.html}とはベクターベースのグラフィックを作成するソフトウェアである.本グループではアプリケーションのアイコンやロゴの作成に使用した.

\section{Discord}
本グループの主な連絡手段としてDiscord\footnote{https://discord.com/}を使用した.プロジェクトの連絡はもちろん,日常生活で起こったことなどをDiscordで共有することによりチームワークの向上を実現することができた.

\section{Docker}
Docker\footnote{https://www.docker.com/}とは開発環境をコンテナとして扱うことができるツールである.本グループではサーバサイドの開発にDockerを用いた.Dockerを用いることにより,デプロイを容易に行うことができた.

\section{FigJam}
FigJam\footnote{https://www.figma.com/}はオンラインでホワイトボードを利用できるツールである.Miro\footnote{https://miro.com/}と同じようなツールだが本グループではFigma\footnote{https://www.figma.com/}を利用しており同じサービス元であるためFigJamへと移行した.

\section{Figma}
Figmaとは,ホームページやアプリケーションなどのワイヤーフレームを作成できるツールである.中間発表スライドの作成,プロトタイプの作成,函館バス訪問の際に使用したスライドの作成,成果発表スライドの作成に使用した.コンポーネントやレスポンシブを意識し,ワイヤーフレームを作成することで,開発がスムーズになるようにした.

\pagebreak
\section{Flutter}
モバイルアプリのフレームワークとしてFlutter\footnote{https://flutter.dev/}を用いた.Android,iOSの両者にサービスを展開する上でKotlin,Swiftそれぞれでネイティブアプリを作成する事も検討したが,グループの規模とメンバーの技術力を考慮して,クロスプラットフォーム開発が可能であるFlutterを用いることに決定した.

\section{GitHub}
GitHub\footnote{https://github.com/}とはソースコードのバージョン管理システムである.このツールを用いることでチームでのアプリ開発を効率的に行うことができる.本グループでは開発サービスでそれぞれリポジトリを作成し,アプリの開発を進めた.

\section{GitHub Projects}
GitHub ProjectsとはGitHubが提供するタスク管理ツールである.GitHub ProjectsにGitHubのIssueを紐付けることにより,タスク管理を行うことができる.本グループではタスクの管理にGitHub Projectsを用いることにより,タスクの進捗状況を把握することができた.

\section{Google Cloud Platform}
Google Cloud Platform\footnote{https://cloud.google.com/}とはGoogleが提供するクラウドサービスである.

\section{Google Drive}
Google Drive\footnote{https://drive.google.com}とはGoogleが提供するオンラインストレージサービスである.Google Driveを用いることにより手軽にファイルの共有を行うことができる.また昨年度のファイル履歴を見返すこともでき,資料作り,プロジェクトの進め方の参考とした.

\section{Miro}
Miroはオンラインでホワイトボードを利用できるツールである.ブレインストーミングなど,メンバーで一斉に意見を書くときに使用した.Miroを使用することにより,メンバーそれぞれの意見,考えを効率よく共有することができた.現在は同じような機能を持ったFigJamを利用している.

\pagebreak
\section{Notion}
本グループでは情報を管理するツールとしてNotion\footnote{https://www.notion.so/}を使用した.議事録,日報,週報,プロダクトバックログをそれぞれページにまとめることにより,情報を見やすくすることができた.

\section{Slack}
本プロジェクトの主な連絡手段としてSlack\footnote{https://slack.com/}を使用した.本グループでは主な連絡手段としてDiscordを利用していたが,プロジェクトではSlackを利用するとなったため使用した.細かくチャンネルを分けることで情報がどこにあるかをわかりやすくすることができた.また,メンションという機能を利用することにより教員,プロジェクトメンバー,TAとそれぞれに通知が行くようにすることで効率の良い意思疎通を行うことができた.

\section{Visual Studio Code}
Visual Studio Code\footnote{https://code.visualstudio.com/}とはマイクロソフトが提供するソースコードエディタである.本グループではコーディングを行う際に使用した.

\section{Xcode}
Xcode\footnote{https://developer.apple.com/jp/xcode/}とはAppleが提供する統合開発環境である.本グループではコーディングエディタとしてVisual Studio Codeを使用したが,iOS端末での動作確認を行う際エミュレータを使用するためXcodeを使用した.


\section{リスクマネジメント}
プロジェクトを行ううえでリスクは必ず存在する.リスクマネジメント\cite{risk}はそのリスクに対してどのように対策するのかを考える.まず初めにプロジェクトメンバーを3グループに分け,それぞれのチームでメンバーが持ち寄った起こりそうなリスクをまとめた.その後それぞれのリスクに対して,そのリスクが被る被害,発生確率,影響度,脅威,対策を考えた.脅威については発生確率,影響度を掛け合わせたものであり,脅威の値が大きいほど対策するべき優先度が高いものとなる.

\bunseki{大津武琉}
