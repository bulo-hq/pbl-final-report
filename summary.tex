\chapter{まとめ}
本グループでは、バス利用者がストレスを抱えずにバスをできるようなアプリ「BuLo(ブーロ)」の開発を行った。最初にブレインストーミングを行い、メンバそれぞれがやりたいことを出し合った。その中で、公共交通機関を利用しづらい問題を解決したいという意見が出た。その意見に対してメンバの多くが共感し、本グループの開発するプロダクトを決定した。次に、プロトタイプの作成を行い、プロジェクトメンバによるユーザテストを行なった。その結果、ターゲットユーザが定まっておらず、本サービスは何のため、誰のためのアプリなのかがわからないことに気がついた。そこで、本サービスについてもう一度考え直すことでターゲットユーザを確立させ、それをもとにプロトタイプを改善した。本サービスでは、函館バス株式会社のデータを利用するため本サービスの紹介を兼ねてデータの使用の許可を得るため、函館バス株式会社を訪問した。その結果、「新しい観点からの機能で良い」というお言葉をいただき、データの使用の許可をいただいた。その後、プロトタイプをもとにクライアントサイド、サーバサイドそれぞれ開発を進めていき、7月7日(金)に行われた中間発表会では、その時点での開発状況を発表し、プロトタイプを用いてデモを行なった。そこで、機能やUIについての指摘をいただき、それをもとに本サービスの改善を行なった。再びクライアントサイド、サーバサイドそれぞれで開発を進めた後、11月3日(金)に行われたはこだて高等教育機関合同研究発表会で本サービスについての発表を行なった。そこでは、情報系を専門としていない、市民や高校生などの一般の方々からも意見や質問をいただくことできた。その後、本サービスの開発を進め、12月8日(金)に行われてた成果発表会にて、本サービスの最終発表を行なった。そこでは、未来大学関係者や函館市民、企業の方々と多くの方に本サービスに興味を持っていただくことができた。また、様々な観点からの意見や質問をいただくことができ、本サービスの改善に繋げることができた。現状では、本サービスの一般リリースを行うことがきていないが、様々な発表を通じて本サービスのニーズがあることを確認できているため、今後も開発を進めて一般リリースを目指していきたい。
\bunseki{大津武琉}
