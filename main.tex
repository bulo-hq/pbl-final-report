\documentclass[openany,11pt,papersize,dvipdfm]{jsbook}
\usepackage[final]{funpro}
\usepackage{graphicx}

\def\hissu{\bgroup\color{red}}
\def\endhissu{\egroup}

\thisYear{2023}

% プロジェクト名
\jProjectName{使ってもらって学ぶフィールド指向システムデザイン 2023}
\eProjectName{Field Oriented System Design Learning by Users' Feedback 2023}

% <プロジェクト番号>-<グループ名>
\ProjectNumber{3-A}

% グループ名
\jGroupName{グループ~A}
\eGroupName{Group~A}

% プロジェクトリーダ
\ProjectLeader{佐々木虎太郎}{Kotaro~Sasaki}

% グループリーダ
\GroupLeader  {及川寛太}{Kanta~Oikawa}

% メンバー数
\SumOfMembers{4}
% グループメンバ
\GroupMember {1}{及川寛太}{Kanta~Oikawa}
\GroupMember {2}{下村蒔里萌}{Marimo~Shimomura}
\GroupMember {3}{大津武琉}{Takeru~Otsu}
\GroupMember {4}{稲田敬介}{Keisuke~Inada}

% 指導教員
\jadvisor{伊藤恵,南部美砂子,奥野拓,元木環,石尾隆}
\eadvisor{Kei~Ito,Misako~Nambu,Taku~Okuno,Tamaki~Motoki,Takashi~Ishio}

% 論文提出日
\jdate{2024年1月17日}
\edate{January~17, 2024}

\begin{document}
%
% 表紙
\maketitle

%======================================================================
%前付け
\frontmatter

% 和文概要
\begin{jabstract} 日本語の概要を書く。
% 和文キーワード
\begin{jkeyword}
キーワード1, キーワード2, キーワード3, キーワード4, キーワード5
\end{jkeyword}
\bunseki{未来太郎}
\end{jabstract}

%英語の概要
\begin{eabstract} Abstract in English. 
% 英文キーワード
\begin{ekeyword}
Keyrods1, Keyword2, Keyword3, Keyword4, Keyword5
\end{ekeyword}
\bunseki{函館花子}
\end{eabstract}

\tableofcontents% 目次

%======================================================================
\mainmatter% 本文のはじまり

%各章の.texファイルをここに並べる
%ファイル名の「.tex」は省略して良い
\chapter{背景}

\bunseki{北海花子}

\section{プロジェクトの立ち上げ}

\bunseki{北海太郎}

\section{プロジェクトの方針}

\bunseki{未来花子}

\section{交通グループ}

\bunseki{未来太郎}

\chapter{到達目標}

\begin{figure}[htbp]
 \centering
 \includegraphics[width=3cm]{images/apple.eps}
 \caption{リンゴ}
 \label{fig:apple}
\end{figure}

\section{本プロジェクトにおける目的}\label{sec:mokuteki}
%1.3節で述べた課題をより具体的に記述する。成果に対して必ず満たすべき条件を含む

地域の特色を生かしたおいしいカレーの作り方が課題。
最終的には、100人中75人以上がおいしいというカレーの詳細なレシピを作ること。
またそのカレーは函館の海産物を用いたものであること。
レシピの手順は、なぜその方法がいいのかも含めて記述されていること。
\bunseki{未来}

\subsection{通常の授業ではなく、プロジェクト学習で行う利点}

本課題では材料に多種多様なものが考えられるが、複数の人数で試作することにより、 
さまざまなバリエーションが試せる。また、味の好みの偏りが少なくなる。 
通常の授業では基本的に個人の知識・技術について講義・演習形式で行われるため、 
共同作業で行うべき作業時間の多いテーマに関しては向かない。

\bunseki{函館}

\subsection{地域との関連性(必要ならば)}

海産物を特徴にしたカレーができると地域の名物料理として売り出せるかも。 
また函館の特産品の売り上げが伸びるかも。

\bunseki{北海}


\section{具体的な手順・課題設定}\label{sec:tejun}
%2.1節で述べた課題を解決するための小課題を手順に分け、その詳細を記述する。 各人への割り当て可能なレベルまで具体化する。なお、以下を必ず含むこと。 
%・このような課題設定に至るプロセス
%・各小課題の解決過程に関連する講義 
%・各小課題の解決過程で用いる既存技術、また習得技術

数多いレシピをもとに効率的に函館特産物を用いたカレーを製作し、コスト面から、 
試作の数を20種類以内に収める目的で、情報収集に力をいれ、以下のように 
手順を設定した。

\begin{enumerate}
\item 従来のカレーレシピ収集(料理本・テレビ・Web)
\par 課題:レシピを共通する部分と異なる部分にわけ、グループ化する。
異なる部分については、それぞれのメリットデメリットを挙げる。

\item 函館特産食品の種類の調査
\par 課題:生産高が多く、一般に特産品として知名度の高いものを調査する。
季節・標準的な値段・一般的な調理法とその調理法により引き出せる味の調査をする。

\item 函館特産食品とカレーとの組み合わせを調べる(過去のレシピの検索)
\par 課題:省略

\item 各材料の下ごしらえのレシピ化。
\par 課題:省略

\item 試作レシピパターンの決定
\par 課題:省略

\item 試作
\par 課題:省略

\item アンケート実施及び解析、改善点の発見
\par 課題:省略

\item 好評な試作パターンについてのバリエーションを設定。
\par 課題:省略

\item アンケート実施及び解析、改善点の発見
(75パーセント以上の好評価を得るまで8-9の繰り返し)
\par 課題:省略
\end{enumerate}

\bunseki{未来}

\section{課題の割り当て}
%2.2節で具体化した各小課題を誰にどのように分担したか、またその理由も含めて記述する。

各人の得意分野及び関連性、時間軸のスケジュールを基準に
以下のように割り当てた。
\bunseki{函館}

\chapter{課題解決のプロセスの概要}
2.2節で具体化した各小課題の解決のプロセスの概要を、各々記述する。

\begin{enumerate}
 \item 従来のカレーレシピ収集(料理本・テレビ・Web) 
\par 解決過程:添付資料(料理本*冊・テレビ番組*本・Web*種類)から
     レシピを収集した。 
\end{enumerate}


その後ほぼ共通する部分を抜き出したところ、*パターンあり、
それらの特徴的な味の変化は付録*参照。以下省略。
\bunseki{北海}

\chapter{課題解決のプロセスの詳細}

\section{各人の課題の概要とプロジェクト内における位置づけ}
%各人の担当課題の概要と、プロジェクト内における役割・位置づけを記述する。

未来花子の担当課題は以下のとおりである。 
\begin{description}
 \item[4月] Webからのレシピ収集・データベース化 。
 \item[5月] レシピの内容のグループ分け。
 \item[6月] 特産品**を含むレシピ検索。
 \item[7--9月]特産品**を含むレシピ考案。
\end{description}

北海花子の担当課題は以下のとおりである。 
\begin{description}
 \item[4月] 草むしり。
 \item[5月] 畑仕事。
 \item[6月] 庭弄り。
\end{description}

\bunseki{未来}

\section{担当課題解決過程の詳細}
%各人の担当課題の解決過程を詳細に記述する新規習得技術を必ず含むこと。

\subsection{未来太郎}
\begin{description}
 \item[4月] Webからのレシピ収集・データベース化 
Webの検索機能を用いて、レシピを検索した。 
材料と手順について、データベースを作成した。 
データベース化には、講義「**」で習得した**の手法を用いた。 
その際、**の技術を習得した。
 \item[5月]まずは草むしりからはじめた。
\end{description}

\bunseki{未来太郎}


\subsection{北海花子}

省略。
\bunseki{北海花子}


\section{担当課題と他の課題の連携内容}
%各人の担当課題とプロジェクト内の他の課題との連携について記述する。

\subsection{未来花子}
自分はWebからのレシピ収集を行ったが、そのデータベースは他の分野から収集した https://ja.overleaf.com/project/5f0efd508f2e5f00015c9b4f
レシピをも入力するものである。データベース構築の際、**の部分を担当した。
以下略。
\bunseki{未来花子}

\subsection{北海花子}
省略。
\bunseki{北海花子}

\chapter{アプリケーションのアイデア}

\section{アプリケーションのコンセプト}
BuLoはバスに乗り遅れたくないひとのためのバスロケーションアプリである.従来のバスロケーションアプリやGoogle Mapsなどの地図アプリとは異なり,バスの位置や遅延情報をリアルタイムにわかりやすく把握することができる.また,“ひとめぼれ”するバスロケーションアプリをめざしている.本グループは”ひとめぼれ”を以下の2つと考える.まず1つ目にアプリのデザインに対する”ひとめぼれ”である.これはアプリを使うきっかけとなるものである.2つ目にアプリ全体を通しての体験への”ひとめぼれ”である.これは,アプリを使い続けるきっかけになると考える.
\bunseki{下村蒔里萌}

\section{搭載している機能}
ああ
\bunseki{下村蒔里萌}
\chapter{結果}

\section{プロジェクトの結果}
%問題点の解決のために製作・考案したものについて記述する。

地域の特色を生かしたおいしいカレーの詳細なレシピを5パターン作った。 
それは、以下のとおりである。

\bunseki{未来}

\section{成果の評価}
%プロジェクト全体の成果について、成果によってどのように上述した課題が解決されたか、成果の効果は当初の想定に沿っているか、残された問題点はあるかを記述する。

成果物のレシピを用いることにより、以下略。

\bunseki{未来花子}



\section{担当分担課題の評価}
%各人の担当課題の成果について、成果によってどのように上述した課題が解決されたか、 要求された役割は果たせたか、残された問題点はあるかを記述する。

\subsection{北海花子}
\begin{description}
 \item[Webからのレシピ収集・データベース化] 
  数多くのデータをデータベース化することによって、
  必要な情報を効率的に検索することができた。 ただし手順・
  材料のデータの解析方法は**の点でデータが重複して得られることがあり、
  その点に関しては改善の余地があると考えられる。 
 \item[ほげほげ]
  ほげはほげであり、ほげほげである。
\end{description}

\bunseki{北海花子}

\subsection{北海太郎}

\bunseki{北海太郎}



% 以降、付録(付属資料)であることを示す
\begin{appendix}

\chapter{新規習得技術}
%課題解決過程に習得した技術について解説する。

\chapter{活用した講義}
%課題解決過程において活用した講義について、講義名・活用内容を記述する。 

\chapter{相互評価}
%課題解決過程で分担し、連携した作業全般について、互いに客観的に評価する。 

\chapter{その他製作物}
%その他成果物をプロジェクトの担当教員の指示に従って添付する。

%付録の終わり
\end{appendix}

%======================================================================
%\backmatter

% 参考文献
\begin{thebibliography}{9}
 \bibitem {ラベル} 著者名. 書籍名. 出版社,  年号.
 \bibitem {A2} ほげほげお. うんたらかんたら,  2003.
\end{thebibliography}

\end{document}
