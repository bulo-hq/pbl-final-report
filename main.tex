\documentclass[openany,11pt,papersize,dvipdfm]{jsbook}
\usepackage[final]{funpro}
\usepackage{graphicx}

\def\hissu{\bgroup\color{red}}
\def\endhissu{\egroup}

\thisYear{2023}

% プロジェクト名
\jProjectName{使ってもらって学ぶフィールド指向システムデザイン 2023}
\eProjectName{Field Oriented System Design Learning by Users' Feedback 2023}

% <プロジェクト番号>-<グループ名>
\ProjectNumber{3-A}

% グループ名
\jGroupName{グループ~A}
\eGroupName{Group~A}

% プロジェクトリーダ
\ProjectLeader{佐々木虎太郎}{Kotaro~Sasaki}

% グループリーダ
\GroupLeader  {及川寛太}{Kanta~Oikawa}

% メンバー数
\SumOfMembers{4}
% グループメンバ
\GroupMember {1}{及川寛太}{Kanta~Oikawa}
\GroupMember {2}{下村蒔里萌}{Marimo~Shimomura}
\GroupMember {3}{大津武琉}{Takeru~Otsu}
\GroupMember {4}{稲田敬介}{Keisuke~Inada}

% 指導教員
\jadvisor{伊藤恵,南部美砂子,奥野拓,元木環,石尾隆}
\eadvisor{Kei~Ito,Misako~Nambu,Taku~Okuno,Tamaki~Motoki,Takashi~Ishio}

% 論文提出日
\jdate{2024年1月17日}
\edate{January~17, 2024}

\begin{document}
%
% 表紙
\maketitle

%======================================================================
%前付け
\frontmatter

% 和文概要
\begin{jabstract} 日本語の概要を書く。
% 和文キーワード
\begin{jkeyword}
キーワード1, キーワード2, キーワード3, キーワード4, キーワード5
\end{jkeyword}
\bunseki{未来太郎}
\end{jabstract}

%英語の概要
\begin{eabstract} Abstract in English. 
% 英文キーワード
\begin{ekeyword}
Keyrods1, Keyword2, Keyword3, Keyword4, Keyword5
\end{ekeyword}
\bunseki{函館花子}
\end{eabstract}

\tableofcontents% 目次

%======================================================================
\mainmatter% 本文のはじまり

\chapter{背景}

\bunseki{北海花子}

\section{プロジェクトの立ち上げ}

\bunseki{北海太郎}

\section{プロジェクトの方針}

\bunseki{未来花子}

\section{交通グループ}

\bunseki{未来太郎}
% 背景
\chapter{プロダクト}

\section{アプリケーションのコンセプト}
BuLoは,バスに乗り遅れたくないがバスを効率的に使いたいひとのためのバスロケーションアプリである.
従来のバスロケーションアプリやGoogle Mapsなどの地図アプリとは異なり,バスの位置や遅延情報をリアルタイムにわかりやすく把握することができる.
また,本サービスは“ひとめぼれ”するバスロケーションアプリをめざしている.本グループは”ひとめぼれ”を以下の2つと考える.
まず1つ目にアプリのデザインに対する”ひとめぼれ”である.これはアプリを使うきっかけとなるものである.
2つ目にアプリ全体を通しての体験への”ひとめぼれ”である.これはアプリを使い続けるきっかけになると考える.
\bunseki{下村蒔里萌}

\section{機能}
\subsection{住所を登録}
    このアプリの対象ユーザは通勤・通学にバスを利用する人であるため,自宅と職場の住所を登録する機能を搭載している.
    対象ユーザの使用するルートは変わらないことが予測できるため,最初に住所を登録をすることで2回目以降は再度住所の検索をする手間を省いている.
    図\ref{fig:feature_register}の左から3枚目の画面では,現在地の住所をサジェストしている.
    住所の入力では,キーワードに対しての予測をサジェストしている.
    \pagebreak
    \begin{figure}
        \centering
        \includegraphics[width=14cm]{images/feature_register.png}
        \caption{住所を登録}
        \label{fig:feature_register}
    \end{figure}
\subsection{Time-Distance ListとRoute View}
    ユーザが住所を登録したあと,ユーザがアプリを開いた際は図\ref{fig:feature_td}の画面を表示する.
    図\ref{fig:feature_td}の左側の画面は,後述するTime-Distance Viewの一覧である.
    これは,バスがユーザの乗車するバス停に到着する順に表示している.
    図\ref{fig:feature_td}の右側の画面は,各Time-Distance Viewをタップした際に表示される画面である.
    タップしたTime-Distance Viewと,ユーザとそのバスの現在地をマップ上に表示している.
    \pagebreak
    \begin{figure}
        \centering
        \includegraphics[width=14cm]{images/feature_td.png}
        \caption{Time-Distance ViewとRoute View}
        \label{fig:feature_td}
    \end{figure}
\subsection{Time-Distance Viewの詳細}
    Time-Distance Viewとは,ユーザの現在地,バスの現在地,バス停の3点を時間的グラフに表したものである.
    これにより,ユーザは「バスに間に合うかどうか」,「バス停でどのくらい待つか」がひとめでわかる.
    実装方法は図\ref{fig:feature_timedistanceview}に示す.
    また,図\ref{fig:feature_timedistanceview}では以下の状況を考えている.
    \pagebreak
    \begin{quote}
        \begin{itemize}
            \item ユーザの目的地は自宅(田屋入口の近く)
            \item 現在地からの最寄りのバス停ははこだて未来大学駅
            \item ユーザが乗るバスは55G
            \item ユーザが自宅に行くまでにかかる料金は220円
            \item バスは5分後にはこだて未来大学駅に到着する
            \item 現在地からはこだて未来大学駅まで徒歩で3分かかる
            \item 自宅には16:25に到着する
        \end{itemize}
    \end{quote}
    \begin{figure}
        \centering
        \includegraphics[width=14cm]{images/feature_timedistanceview.png}
        \caption{Time-Distance View}
        \label{fig:feature_timedistanceview}
    \end{figure}
    \pagebreak
    図\ref{fig:feature_timedistanceview2}では,想定されるTime-Distance Viewの例を示している.
    図\ref{fig:feature_timedistanceview2}の左から,「ユーザがバスに安心して乗れる状態」
    「ユーザがバス停でかなり待つことが予想できるため,バス停以外の場所で時間をつぶすことができる」
    「ユーザよりバスの方が先にバス停に到着するが,走ることでバスに乗ることができる」状態である.
    \pagebreak
    \begin{figure}
        \centering
        \includegraphics[width=14cm]{images/feature_timedistanceview2.png}
        \caption{Time-Distance Viewの例}
        \label{fig:feature_timedistanceview2}
    \end{figure}
\bunseki{下村蒔里萌}

\section{デザインシステム}
    
\bunseki{下村蒔里萌}

\section{システムの構成}
\subsection{アーキテクチャ図}
\subsection{使用した技術}
\bunseki{}% プロダクト
\include{process}% 活動
\chapter{今後の予定}
今後の予定は、1つ目に、複数事業者に対応させることを目指す。現状のバージョンでは函館バスのみしか対応していない。これを拡張し、函館市電やJR、また函館市以外のバス事業者にも対応させる。2つ目に乗り換えに対応させることを目指す。バスを利用する上で目的地まで行くのに乗り換えが起こることは多くある。また、すでに「乗り換えに対応して欲しい」という指摘が多く寄せられているので乗り換えに対応させることをを目指す。3つ目にパフォーマンスの向上を図る。現状のバージョンでは、住所を検索する際や、路線を検索する際、結果が出るまでに時間がかかってしまっている。本サービスは「ひとめぼれ」をコンセプトとしているため、このレスポンスの遅さを改善することを目指す。4つ目に、Route Viewでのルートの表示を目指す。ユーザのルートの表示の目処は立っている。しかし、函館バスが提供しているデータの中に、ルートの情報がないため、バスのルートの表示の目処が立っていない。そこでどのようにしてルートを表示するかを考え、実装を行う。最後に、調査に基づいてUI/UXの改善を行う。すでに中間発表会や成果発表会でUI/UXの改善点を指摘された。それらの指摘に加えて、一般リリースをしたあとの、ユーザのフィードバックをもとに改善を行い、より使いやすいアプリにしていく。
\bunseki{大津武琉}% ロードマップ
\chapter{学び}

\section{個人の学び}
\subsection{稲田敬介}
\bunseki{稲田敬介}

\subsection{及川寛太}
\bunseki{及川寛太}

\subsection{大津武琉}
\bunseki{大津武琉}

\subsection{下村蒔里萌}
1年間のプロジェクト学習を経て,多くのことを学ぶことができた.
まず,アジャイル開発のスクラム手法より,短期間でより良いプロダクトを作り上げる意識や感覚を学べた.
対象ユーザにとってより良いサービスを提供するために,短期間でリリースを行い,フィードバックを活かしていくことの重要性を理解できた.
また,スクラム手法を実現するためにはメンバーとのコミュニケーションや進捗報告が重要であることを学んだ.
各々がスプリントゴールを達成するには,個人の持つ時間や能力に合わせてタスク分配をすることが重要な上,予定より作業が遅れている場合には直ちに軌道修正が求められる.
そのためにも,円滑にコミュニケーションを行うにはどうすれば良いか,どのように質問や進捗報告を行えば良いかを学ぶことができた.
次に,
\bunseki{下村蒔里萌}

% プロジェクトを通しての学び



% 以降、付録(付属資料)であることを示す
\begin{appendix}

\chapter{使用した技術・ツール・知識}

\section{Adobe Illustrator}
Adobe Illustrator\footnote{https://www.adobe.com/jp/products/illustrator.html}は,ベクターベースのグラフィックを作成するソフトウェアである.本グループではアプリケーションのアイコンやロゴの作成に使用した.

\section{Discord}
Discord\footnote{https://discord.com/}は,コミュニケーションツールである.本グループの主な連絡手段として使用した.プロジェクトに関わる連絡はもちろんのこと,日常生活で起こったことなどをDiscordで共有することによりチームワークの向上を実現することができた.

\section{Docker}
Docker\footnote{https://www.docker.com/}は,開発環境をコンテナとして扱うことができるツールである.本グループではサーバサイドの開発にDockerを用いた.Dockerを用いることにより,デプロイを容易に行うことができた.

\section{FigJam}
FigJam\footnote{https://www.figma.com/}はオンラインでホワイトボードを利用できるツールである.Miro\footnote{https://miro.com/}で同じような機能を使うことができるが,Figma\footnote{https://www.figma.com/}の利用を開始したため,FigJamへと移行した.

\section{Figma}
Figmaは,ホームページやアプリケーションなどのワイヤーフレームを作成できるツールである.中間発表スライドの作成,プロトタイプの作成,函館バス訪問の際に使用したスライドの作成,成果発表スライドの作成に使用した.コンポーネントやレスポンシブを意識し,ワイヤーフレームを作成することで,開発がスムーズになるようにした.

\section{Flutter}
Flutter\footnote{https://flutter.dev/}は,単一のコードベースから,複数のプラットフォームに向けたアプリケーションを構築できるフレームワークである.
UI (ユーザインタフェース) を複数のプラットフォームで統一できることなどのメリットがある.

\section{GitHub}
GitHub\footnote{https://github.com/}は,ソースコードのバージョン管理システムである.このツールを用いることでチームでのアプリ開発を効率的に行うことができる.本グループでは開発サービスでそれぞれリポジトリを作成し,アプリの開発を進めた.

\section{GitHub Projects}
GitHub Projectsは,GitHubが提供するタスク管理ツールである.GitHub ProjectsにGitHubのIssueを紐付けることにより,タスク管理を行うことができる.本グループではタスクの管理にGitHub Projectsを用いることにより,タスクの進捗状況を把握することができた.

\section{Google Cloud Platform}
Google Cloud Platform\footnote{https://cloud.google.com/}は,Googleが運営するPaaS (Platform as a Service) である.

\section{Google Drive}
Google Drive\footnote{https://drive.google.com/}は,Googleが提供するオンラインストレージサービスである.Google Driveを用いることにより手軽にファイルの共有を行うことができる.また昨年度のファイル履歴を見返すこともでき,資料作り,プロジェクトの進め方の参考とした.

\section{GraphQL}
GraphQL\footnote{https://graphql.org/}は,API用のクエリ言語であり,そのクエリを実行するためのランタイムである.
メリットは,クライアントがAPIに対して必要なリソースだけを要求することができること,可読性の高いスキーマにより開発におけるコミュニケーションコストを下げることができることである.
デメリットは,REST APIやgRPCと比較して,実行速度が遅いことである.

\section{gRPC}
gRPC\footnote{https://grpc.io/}は,ProtocolBuffers\footnote{https://protobuf.dev/}を用いて通信する,RPC (Remote Procedure Call) フレームワークである.

\section{Miro}
Miroは,オンラインでホワイトボードを利用できるツールである.ブレインストーミングなど,メンバーで一斉に意見を書くときに使用した.メンバーそれぞれの意見,考えを効率よく共有することができた.

\section{Notion}
本グループでは情報を管理するツールとしてNotion\footnote{https://www.notion.so/}を使用した.議事録,日報,週報,プロダクトバックログをそれぞれページにまとめることにより,情報を見やすくすることができた.

\section{PostgreSQL}
PostgreSQL\footnote{https://www.postgresql.org/}は,開発において使用したDBMS (Database Management System) である.
SQLを用いてデータの操作を行う,RDB (Relational Database) である.
メリットは,拡張性,信頼性が高いことである.
デメリットは,MySQLと比較して,実行速度が遅いことである.

\section{Slack}
Slack\footnote{https://slack.com/}は,コミュニケーションツールである.本グループでは主な連絡手段としてDiscordを利用していたが,本プロジェクトの主な連絡手段として使用した.細かくチャンネルを分けることで情報がどこにあるかをわかりやすくすることができた.また,メンションという機能を利用することにより教員,プロジェクトメンバー,TAとそれぞれに通知が届くようにすることで効率的な意思疎通を図ることができた.

\section{Visual Studio Code}
Visual Studio Code\footnote{https://code.visualstudio.com/}は,マイクロソフトが提供するソースコードエディタである.

\section{Xcode}
Xcode\footnote{https://developer.apple.com/jp/xcode/}は,Appleが提供する統合開発環境である.本グループではコードエディタとしてVisual Studio Codeを使用したが,iOS端末での動作確認を行う際には,エミュレータを使用するために使用した.


\section{リスクマネジメント}
プロジェクトを行ううえでリスクは必ず存在する.リスクマネジメント\cite{risk}はそのリスクに対してどのように対策するのかを考える.まず初めにプロジェクトメンバーを3グループに分け,それぞれのチームでメンバーが持ち寄った起こりそうなリスクをまとめた.その後それぞれのリスクに対して,そのリスクが被る被害,発生確率,影響度,脅威,対策を考えた.脅威については発生確率,影響度を掛け合わせたものであり,脅威の値が大きいほど対策するべき優先度が高いものとなる.

\bunseki{大津武琉}

\chapter{資料}
\section{成果発表スライド}
\begin{figure}[H]
    \centering
    \includegraphics[width=14cm]{images/slide0.png}
    \label{fig:slide0}
\end{figure}
\begin{figure}[H]
    \includegraphics[width=14cm]{images/slide1.png}
    \label{fig:slide1}
\end{figure}
\begin{figure}[H]
    \includegraphics[width=14cm]{images/slide2.png}
    \label{fig:slide2}
\end{figure}
\begin{figure}[H]
    \includegraphics[width=14cm]{images/slide3.png}
    \label{fig:slide3}
\end{figure}
\begin{figure}[H]
    \includegraphics[width=14cm]{images/slide4.png}
    \label{fig:slide4}
\end{figure}
\begin{figure}[H]
    \includegraphics[width=14cm]{images/slide5.png}
    \label{fig:slide5}
\end{figure}
\begin{figure}[H]
    \includegraphics[width=14cm]{images/slide6.png}
    \label{fig:slide6}
\end{figure}
\begin{figure}[H]
    \includegraphics[width=14cm]{images/slide7.png}
    \label{fig:slide7}
\end{figure}
\begin{figure}[H]
    \includegraphics[width=14cm]{images/slide8.png}
    \label{fig:slide8}
\end{figure}
\begin{figure}[H]
    \includegraphics[width=14cm]{images/slide9.png}
    \label{fig:slide9}
\end{figure}
\begin{figure}[H]
    \includegraphics[width=14cm]{images/slide10.png}
    \label{fig:slide10}
\end{figure}

\section{成果発表ポスター}
\begin{figure}[H]
    \centering
    \includegraphics[width=14cm]{images/poster.png}
\end{figure}
\bunseki{下村蒔里萌}

\end{appendix}

%======================================================================
%\backmatter

% 参考文献
\begin{thebibliography}{9}
 \bibitem {ラベル} 著者名. 書籍名. 出版社,  年号.
 \bibitem {A2} ほげほげお. うんたらかんたら,  2003.
\end{thebibliography}

\end{document}
