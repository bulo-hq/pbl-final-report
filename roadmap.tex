\chapter{今後の予定}
今後の予定は,1つ目に,複数事業者に対応させることを目指す.現状のバージョンでは函館バスのみしか対応していない.これを拡張し,函館市電やJR,また函館市以外のバス事業者にも対応させる.2つ目に乗り換えに対応させることを目指す.バスを利用する上で目的地まで行くのに乗り換えが起こることは多くある.また,すでに「乗り換えに対応して欲しい」という指摘が多く寄せられているので乗り換えに対応させることをを目指す.3つ目にパフォーマンスの向上を図る.現状のバージョンでは,住所を検索する際や,路線を検索する際,結果が出るまでに時間がかかってしまっている.本サービスは「ひとめぼれ」をコンセプトとしているため,このレスポンスの遅さを改善することを目指す.4つ目に,Route Viewでのルートの表示を目指す.ユーザのルートの表示の目処は立っている.しかし,函館バスが提供しているデータの中に,ルートの情報がないため,バスのルートの表示の目処が立っていない.そこでどのようにしてルートを表示するかを考え,実装を行う.最後に,調査に基づいてUI/UXの改善を行う.すでに中間発表会や成果発表会でUI/UXの改善点を指摘された.それらの指摘に加えて,一般リリースをしたあとの,ユーザのフィードバックをもとに改善を行い,より使いやすいアプリにしていく.
\bunseki{大津武琉}