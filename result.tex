\chapter{結果}

\section{プロジェクトの結果}
%問題点の解決のために製作・考案したものについて記述する。

地域の特色を生かしたおいしいカレーの詳細なレシピを5パターン作った。 
それは、以下のとおりである。

\bunseki{未来}

\section{成果の評価}
%プロジェクト全体の成果について、成果によってどのように上述した課題が解決されたか、成果の効果は当初の想定に沿っているか、残された問題点はあるかを記述する。

成果物のレシピを用いることにより、以下略。

\bunseki{未来花子}



\section{担当分担課題の評価}
%各人の担当課題の成果について、成果によってどのように上述した課題が解決されたか、 要求された役割は果たせたか、残された問題点はあるかを記述する。

\subsection{北海花子}
\begin{description}
 \item[Webからのレシピ収集・データベース化] 
  数多くのデータをデータベース化することによって、
  必要な情報を効率的に検索することができた。 ただし手順・
  材料のデータの解析方法は**の点でデータが重複して得られることがあり、
  その点に関しては改善の余地があると考えられる。 
 \item[ほげほげ]
  ほげはほげであり、ほげほげである。
\end{description}

\bunseki{北海花子}

\subsection{北海太郎}

\bunseki{北海太郎}
