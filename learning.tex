\chapter{学び}

\section{個人の学び}
\subsection{稲田敬介}
\bunseki{稲田敬介}

\subsection{及川寛太}
\bunseki{及川寛太}

\subsection{大津武琉}
\bunseki{大津武琉}

\subsection{下村蒔里萌}
1年間のプロジェクト学習を経て,多くのことを学ぶことができた.
まず,アジャイル開発のスクラム手法より,短期間でより良いプロダクトを作り上げる意識や感覚を学べた.
対象ユーザにとってより良いサービスを提供するために,短期間でリリースを行い,フィードバックを活かしていくことの重要性を理解できた.
また,スクラム手法を実現するためにはメンバーとのコミュニケーションや進捗報告が重要であることを学んだ.
各々がスプリントゴールを達成するには,個人の持つ時間や能力に合わせてタスク分配をすることが重要な上,予定より作業が遅れている場合には直ちに軌道修正が求められる.
そのためにも,円滑にコミュニケーションを行うにはどうすれば良いか,どのように質問や進捗報告を行えば良いかを学ぶことができた.
次に,
\bunseki{下村蒔里萌}

