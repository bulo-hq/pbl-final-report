\chapter{プロダクト}

\section{アプリケーションのコンセプト}
BuLoは,バスに乗り遅れたくないがバスを効率的に使いたいひとのためのバスロケーションアプリである.
従来のバスロケーションアプリやGoogle Mapsなどの地図アプリとは異なり,バスの位置や遅延情報をリアルタイムにわかりやすく把握することができる.
また,本サービスは“ひとめぼれ”するバスロケーションアプリをめざしている.我々は”ひとめぼれ”を以下の2つと考える.
まず1つ目にアプリのデザインに対する”ひとめぼれ”である.これは本サービスを使うきっかけとなるものである.
2つ目にアプリ全体を通しての体験への”ひとめぼれ”である.これは本サービスを使い続けるきっかけになると考える.
\bunseki{下村蒔里萌}

\section{機能}
\subsection{住所を登録}
    本サービスは自宅と職場の住所を登録する機能を搭載している.
    対象ユーザは通勤・通学にバスを利用する人であり,使用するルートは変わらないことが予測できるため,最初に住所を登録をすることで2回目以降は再度住所の検索をする手間を省いている.
    図\ref{fig:feature_register}の左から3枚目の画面では,現在地の住所をサジェストしている.
    住所の入力では,キーワードに対しての予測をサジェストしている.
    \begin{figure}
        \centering
        \includegraphics[width=14cm]{images/feature_register.png}
        \caption{住所を登録}
        \label{fig:feature_register}
    \end{figure}
\subsection{Time-Distance ListとRoute View}
    ユーザが住所を登録したあと,ユーザがアプリを開いた際は図\ref{fig:feature_td}の画面を表示する.
    図\ref{fig:feature_td}の左側の画面は,後述するTime-Distance Viewの一覧である.
    一覧は,バスがユーザの乗車するバス停に到着する順に表示される.
    図\ref{fig:feature_td}の右側の画面は,各Time-Distance Viewをタップした際に表示される画面である.
    ここではタップしたTime-Distance Viewと,ユーザとそのバスの現在地がマップ上に表示される.
    \begin{figure}
        \centering
        \includegraphics[width=14cm]{images/feature_td.png}
        \caption{Time-Distance ViewとRoute View}
        \label{fig:feature_td}
    \end{figure}
\subsection{Time-Distance Viewの詳細}
    Time-Distance Viewとは,ユーザの現在地・バスの現在地・バス停の3点を時間的グラフに表したものである.
    これにより,ユーザは「バスに間に合うかどうか」,「バス停でどのくらい待つか」がひとめでわかる.
    実装方法は図\ref{fig:feature_timedistanceview}に示す.
    また,図\ref{fig:feature_timedistanceview}では以下の状況を考えている.
    \begin{quote}
        \begin{itemize}
            \item ユーザの目的地は自宅(田屋入口の近く)
            \item 現在地からの最寄りのバス停ははこだて未来大学駅
            \item ユーザが乗るバスは55G
            \item ユーザが自宅に行くまでにかかる料金は220円
            \item バスは5分後にはこだて未来大学駅に到着する
            \item 現在地からはこだて未来大学駅まで徒歩で3分かかる
            \item 自宅には16:25に到着する
        \end{itemize}
    \end{quote}
    \begin{figure}
        \centering
        \includegraphics[width=14cm]{images/feature_timedistanceview.png}
        \caption{Time-Distance View}
        \label{fig:feature_timedistanceview}
    \end{figure}
    図\ref{fig:feature_timedistanceview2}では,想定されるTime-Distance Viewの例を示している.
    図\ref{fig:feature_timedistanceview2}の左から,「ユーザがバスに安心して乗れる状態」
    「ユーザがバス停でかなり待つことが予想できるため,バス停以外の場所で時間をつぶすことができる」
    「ユーザよりバスの方が先にバス停に到着するが,走ることでバスに乗ることができる」状態である.
    \begin{figure}
        \centering
        \includegraphics[width=14cm]{images/feature_timedistanceview2.png}
        \caption{Time-Distance Viewの例}
        \label{fig:feature_timedistanceview2}
    \end{figure}
\bunseki{下村蒔里萌}

\section{デザインシステム}
    本アプリは,簡潔で見やすく情報量の絞られたデザインを目指している.
    アジャイル開発を行う上で,手戻りの少ない一貫性のあるUI改善を行うため,また,エンジニアとデザイナーの共通認識を図るために,
    以下のデザインシステムを設定した.
\subsection{Colors}
    本サービスのカラーパレットは図\ref{fig:feature_colors}のように設定した.
    16進数のカラーコード(例:\#FF3B30)に具体的な値を与え使いやすくしたプリミティブトークン(例:red)と,
    特定の用途別に定義したセマンティックトークン(例:alert)を定義している.
    Figma上でのプロトタイプや実装では,セマンティックトークンを使用した.
    \begin{figure}
        \centering
        \includegraphics[width=10cm]{images/colors.png}
        \caption{デザインシステム Colors}
        \label{fig:feature_colors}
    \end{figure}
\subsection{Typographies}
    本サービスのフォントは図\ref{fig:typographies}のように設定した.
    \begin{figure}
        \centering
        \includegraphics[width=10cm]{images/typographies.png}
        \caption{デザインシステム Typographies}
        \label{fig:typographies}
    \end{figure}
\subsection{Icons}
    本サービスのアイコンは図\ref{fig:icons}のように設定した.
    主にMaterial Symbols and Icons\footnote{https://fonts.google.com/icons}を使用し,足りないものはPictogrammers\footnote{https://pictogrammers.com/}を使用した.
        \begin{figure}
            \centering
            \includegraphics[width=10cm]{images/icons.png}
            \caption{デザインシステム Icons}
            \label{fig:icons}
        \end{figure}
\subsection{Shapes and Others}
    本サービスの形状とその他のデザインは図\ref{fig:shapes}のように設定した.
    ElavationやBlurは2つのレイヤー間のz軸上の深度を表す.
    インターフェイス上の最も上位の要素をより強調することで、アクションの重要度を伝える.
    Cornersは角丸を表し,4, 8, 16と3種類の数字を用意し、コンポーネントの短辺に合わせて角丸の数値を可変する.
    角丸を使用した図形の中に、角丸を使用した図形位がある場合は、
    外側の図形の角丸=内側の図形の角丸+Paddingとする.
    \begin{figure}
        \centering
        \includegraphics[width=10cm]{images/shapes.png}
        \caption{デザインシステム Shapes and Others}
        \label{fig:shapes}
    \end{figure}
\bunseki{下村蒔里萌}

\section{システムの構成}
本サービスでは,マイクロサービスアーキテクチャを採用している.クライアントは,BFF (Backend for Frontend) を介して,バックエンドのマイクロサービスと通信する.
バックエンドのマイクロサービス間は,gRPCを用いて通信する.アーキテクチャ図を図\ref{fig:architecture}に示す.
\begin{figure}
    \centering
    \includegraphics[width=14cm]{images/architecture_diagram.png}
    \caption{アーキテクチャ図}
    \label{fig:architecture}
\end{figure}
\bunseki{及川寛太}

\section{使用した技術}
\subsection{Flutter}
    Flutter\footnote{https://flutter.dev/}は,単一のコードベースから,複数のプラットフォームに向けたアプリケーションを構築できるフレームワークである.
    メリットは,UI (ユーザインタフェース) を複数のプラットフォームで統一できることである.
    デメリットは,各プラットフォームの最新バージョンに対応するまでに時間がかかるというデメリットがある.
    当初は,iOSアプリをSwiftで,AndroidアプリをKotlinで開発する予定であったが,開発期間が短いことや人数が少ないことから,1つのコードで2つのプラットフォームに対応できる,Flutterを採用した.
\bunseki{及川寛太}

\subsection{GraphQL}
    GraphQL\footnote{https://graphql.org/}は,API用のクエリ言語であり,そのクエリを実行するためのランタイムである.
    メリットは,クライアントがAPIに対して必要なリソースだけを要求することができること,可読性の高いスキーマにより開発におけるコミュニケーションコストを下げることができることである.
    デメリットは,REST APIやgRPCと比較して,実行速度が遅いことである.
\bunseki{及川寛太}

\subsection{PostgreSQL}
    開発において使用したDBMS (Database Management System) である\footnote{https://www.postgresql.org/}.
    SQLを用いてデータの操作を行う,RDB (Relational Database) である.
    メリットは,拡張性,信頼性が高いことである.
    デメリットは,MySQLと比較して,実行速度が遅いことである.
\bunseki{及川寛太}
