\chapter{活用した講義}
\section{アジャイルワークショップ}
アジャイルコーチの永瀬氏に,本プロジェクトで用いるチーム開発の手法であるアジャイル開発について講義をしていただいた.まず,本ワークショップの前に事前知識を得るため,アジャイル開発についてのビデオを視聴した.その知識をもとに永瀬氏からアジャイル開発とはなにか,アジャイル開発で用いられるスクラムとはなにかを学んだ.また,最後に講義を受けているメンバーを5人ずつのチームにし,スクラムについてのクイズ大会を行った.これにより,楽しみながらスクラムについて学ぶことができ,スクラムについての共通認識を持つことができた.

\section{フィールドワーク講座}
本プロジェクトの担当教員でもある,南部美砂子准教授,元木環准教授に講義をしていただき,フィールドワークに関する基本的な考え方,ありがちな間違いをご教授いただいた.収束的なフィールドワークでは,思い込みや認識の狭さなどを超えられないため,フィールドワークには問題を探しにいくのではないこと,また,人々の行為や相互行為には必ず理由や動機が存在し,それらを理解することをフィールドワークの目的とすることを学んだ.

\bunseki{稲田敬介}
